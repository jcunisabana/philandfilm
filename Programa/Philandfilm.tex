\documentclass[]{article}
\usepackage{hyperref}
\usepackage[]{fontspec}
\usepackage{graphicx}


% \usepackage{lmodern}

 % \usepackage{amssymb,amsmath}
% \usepackage{ifxetex,ifluatex}
% \usepackage{fixltx2e} % provides \textsubscript
% \ifnum 0\ifxetex 1\fi\ifluatex 1\fi=0 % if pdftex
%   
% \else % if luatex or xelatex
%   \ifxetex
%     \usepackage{mathspec}
%   \else
%     \usepackage{fontspec}
%   \fi
%   \defaultfontfeatures{Ligatures=TeX,Scale=MatchLowercase}
% \fi
% % use upquote if available, for straight quotes in verbatim environments
% \IfFileExists{upquote.sty}{\usepackage{upquote}}{}
% % use microtype if available
% \IfFileExists{microtype.sty}{%
% \usepackage[]{microtype}
% \UseMicrotypeSet[protrusion]{basicmath} % disable protrusion for tt fonts
% }{}
% \PassOptionsToPackage{hyphens}{url} % url is loaded by hyperref
% \usepackage[unicode=true]{hyperref}
% \hypersetup{
%             pdfborder={0 0 0},
%             breaklinks=true}
% \urlstyle{same}  % don't use monospace font for urls
\usepackage{longtable,booktabs}
% Fix footnotes in tables (requires footnote package)
\IfFileExists{footnote.sty}{\usepackage{footnote}\makesavenoteenv{long table}}{}
\usepackage{graphicx,grffile}
\makeatletter
\def\maxwidth{\ifdim\Gin@nat@width>\linewidth\linewidth\else\Gin@nat@width\fi}
\def\maxheight{\ifdim\Gin@nat@height>\textheight\textheight\else\Gin@nat@height\fi}
\makeatother
% Scale images if necessary, so that they will not overflow the page
% margins by default, and it is still possible to overwrite the defaults
% using explicit options in \includegraphics[width, height, ...]{}
\setkeys{Gin}{width=\maxwidth,height=\maxheight,keepaspectratio}
\IfFileExists{parskip.sty}{%
\usepackage{parskip}
}{% else
\setlength{\parindent}{0pt}
\setlength{\parskip}{6pt plus 2pt minus 1pt}
}
\setlength{\emergencystretch}{3em}  % prevent overfull lines
\providecommand{\tightlist}{%
  \setlength{\itemsep}{0pt}\setlength{\parskip}{0pt}}
\setcounter{secnumdepth}{0}
% Redefines (sub)paragraphs to behave more like sections
\ifx\paragraph\undefined\else
\let\oldparagraph\paragraph
\renewcommand{\paragraph}[1]{\oldparagraph{#1}\mbox{}}
\fi
\ifx\subparagraph\undefined\else
\let\oldsubparagraph\subparagraph
\renewcommand{\subparagraph}[1]{\oldsubparagraph{#1}\mbox{}}
\fi

% set default figure placement to htbp
\makeatletter
\def\fps@figure{htbp}
\makeatother


%\newcommand{\fuenteprincipal}{Hoefler Text}
\newcommand{\fuenteprincipal}{Adobe Jenson Pro}
\setmainfont{\fuenteprincipal}
\setromanfont{\fuenteprincipal}
\newfontfamily\headingfont[]{\fuenteprincipal}


%para bilbio
\usepackage{polyglossia,csquotes}
\setdefaultlanguage{spanish}
\usepackage[backend=biber, style=apa, refsection=subsection]{biblatex}
\DeclareLanguageMapping{spanish}{spanish-apa}
\addbibresource{/Users/juanespejo/Documents/Latex/BibTex/hume.bib}
\defbibheading{bibliography}{\section*{Bibliografía}}
\usepackage[left=2.5 cm,top=3cm,right=2.5cm, bottom=2.5cm]{geometry}



\begin{document}
{ \huge SFM: Hume}\label{seminario-de-filosofuxeda-moderna-hume}

{ \large \emph{Teoría del conocimiento y teoría moral}}
\subsection{Descripción del
seminario}\label{descripciuxf3n-del-seminario}

El objetivo central de esta asignatura es desarrollar las competencias
filosóficas del estudiante por medio del estudio del \emph{Tratado de la
naturaleza humana} del filósofo escocés David Hume. El desarrollo de
estas competencias se buscará a través de la discusión oral y escrita de
dos grandes áreas de la filosofía de Hume: la teoría del conocimiento y
la teoría moral. Para esto, se utilizará la metodología de seminario
alemán: en cada sesión un estudiante estará encargado de liderar la
discusión de una sección del \emph{Tratado} previamente asignada o de
una lectura de la bibliografía secundaria. La sesión girará en torno a
una relatoría escrita por el estudiante encargado que será leída con
anterioridad y discutida en clase por todos los asistentes. Al final del
semestre, el estudiante consolidará lo aprendido en un ensayo
argumentativo en el que desarrolle a mayor profundidad alguno de los
elementos discutidos a lo largo del
semestre.

\emph{Docente:} \href{}{Juan Camilo Espejo-Serna}

\emph{Correo electrónico:} juan.espejo1@unisabana.edu.co

\emph{Horario y salón:} Lunes 10:00am - 1:00pm. AULA A201

\emph{Página web del curso:} http://jcunisabana.github.io/hume

\subsection{Objetivos esperados de
aprendizaje}\label{objetivos-esperados-de-aprendizaje}

\begin{itemize}
\item
  El estudiante lee críticamente, analiza e interpreta la teoría del
  conocimiento de David Hume.
\item
  El estudiante lee críticamente, analiza e interpreta la teoría moral
  de David Hume.
\item
  El estudiante planea y elabora textos interpretativos y argumentativos
  acerca de las teorías de David Hume.
\item
  El estudiante utiliza TIC para apoyar el estudio filosófico de la obra
  de David Hume.
\end{itemize}

\subsection{Evaluación}\label{evaluaciuxf3n}

Se evaluará a los estudiantes por medio de controles de lectura,
protocolos, relatorías, y un ensayo argumentativo.

\textbf{Toda} entrega tarde injustificada verá la nota disminuida en 0.5
unidades por cada día tarde.

\begin{longtable}[]{@{}clc@{}}
\toprule
Corte & Actividad & Valor Porcentual\tabularnewline
\midrule
\endhead
1 & Controles de lectura & 15\%\tabularnewline
1 & Reseña & 15\%\tabularnewline
2 & Protocolo & 15\%\tabularnewline
2 & Relatoría & 15\%\tabularnewline
3 & Abstract del ensayo argumentativo & 10\%\tabularnewline
3 & Ensayo argumentativo & 30\%\tabularnewline
\bottomrule
\end{longtable}

\paragraph{Controles de lectura}\label{controles-de-lectura}

Para cada lectura asignada, los estudiantes deben escribir un texto
corto con la tesis principal, tres afirmaciones/presuposiciones del
texto y tres preguntas/desafíos al texto. Los controles deberán ser
subidos a la plataforma virtual a más tardar el \emph{viernes a las 7 de
la mañana}. Todos los estudiantes empiezan con 5.0 en esta nota; por
cada vez que no se participe dentro del rango de tiempo especificado, la
nota será disminuida de acuerdo con los siguientes parámetros: primera
vez: -0.5; segunda vez: -1.0; tercera vez: -1.5; cuarta vez: -2.0.

\paragraph{Reseña}\label{reseuxf1a}

Resumen argumentativo: debe presentar el punto central del texto,
explicarlo y señalar las razones principales ofrecidas en su favor. En
clase se ofrecerán las instrucciones más precisas cuando llegue el
momento.

\paragraph{Protocolo}\label{protocolo}

Recuento de una sesión. Debe incluir los puntos centrales tanto de la
relatoría como de la discusión y  procurar seguir las mismas
instrucciones dadas para la reseña de un texto escrito. En clase se
ofrecerán las instrucciones más precisas.

\paragraph{Relatoría}\label{relatoruxeda}

Escrito para ser presentado oralmente en clase. Debe incluir una reseña
del texto asignado y presentar puntos para la discusión. En clase se
ofrecerán las instrucciones más precisas.

\paragraph{Abstract del ensayo
argumentativo}\label{abstract-del-ensayo-argumentativo}

Un texto argumentativo de mínimo 400 y máximo 500 palabras en donde se
presente un resumen de la tesis principal y la estrategia argumentativa
que usará.

\paragraph{Ensayo argumentativo}\label{ensayo-argumentativo}

Un texto argumentativo de 4-6 páginas en donde se defienda una tesis
filosófica. Se debe hacer uso de literatura secundaria de acuerdo con
las normas de citación. En clase se ofrecerán las instrucciones más
precisas cuando llegue el momento.

\nocite{*}
\printbibliography
\end{document}
