\documentclass[]{article}
\usepackage{hyperref}
\usepackage[]{fontspec}
\usepackage{graphicx}


% \usepackage{lmodern}

 % \usepackage{amssymb,amsmath}
% \usepackage{ifxetex,ifluatex}
% \usepackage{fixltx2e} % provides \textsubscript
% \ifnum 0\ifxetex 1\fi\ifluatex 1\fi=0 % if pdftex
%   
% \else % if luatex or xelatex
%   \ifxetex
%     \usepackage{mathspec}
%   \else
%     \usepackage{fontspec}
%   \fi
%   \defaultfontfeatures{Ligatures=TeX,Scale=MatchLowercase}
% \fi
% % use upquote if available, for straight quotes in verbatim environments
% \IfFileExists{upquote.sty}{\usepackage{upquote}}{}
% % use microtype if available
% \IfFileExists{microtype.sty}{%
% \usepackage[]{microtype}
% \UseMicrotypeSet[protrusion]{basicmath} % disable protrusion for tt fonts
% }{}
% \PassOptionsToPackage{hyphens}{url} % url is loaded by hyperref
% \usepackage[unicode=true]{hyperref}
% \hypersetup{
%             pdfborder={0 0 0},
%             breaklinks=true}
% \urlstyle{same}  % don't use monospace font for urls
\usepackage{longtable,booktabs}
% Fix footnotes in tables (requires footnote package)
\IfFileExists{footnote.sty}{\usepackage{footnote}\makesavenoteenv{long table}}{}
\usepackage{graphicx,grffile}
\makeatletter
\def\maxwidth{\ifdim\Gin@nat@width>\linewidth\linewidth\else\Gin@nat@width\fi}
\def\maxheight{\ifdim\Gin@nat@height>\textheight\textheight\else\Gin@nat@height\fi}
\makeatother
% Scale images if necessary, so that they will not overflow the page
% margins by default, and it is still possible to overwrite the defaults
% using explicit options in \includegraphics[width, height, ...]{}
\setkeys{Gin}{width=\maxwidth,height=\maxheight,keepaspectratio}
\IfFileExists{parskip.sty}{%
\usepackage{parskip}
}{% else
\setlength{\parindent}{0pt}
\setlength{\parskip}{6pt plus 2pt minus 1pt}
}
\setlength{\emergencystretch}{3em}  % prevent overfull lines
\providecommand{\tightlist}{%
  \setlength{\itemsep}{0pt}\setlength{\parskip}{0pt}}
\setcounter{secnumdepth}{0}
% Redefines (sub)paragraphs to behave more like sections
\ifx\paragraph\undefined\else
\let\oldparagraph\paragraph
\renewcommand{\paragraph}[1]{\oldparagraph{#1}\mbox{}}
\fi
\ifx\subparagraph\undefined\else
\let\oldsubparagraph\subparagraph
\renewcommand{\subparagraph}[1]{\oldsubparagraph{#1}\mbox{}}
\fi

% set default figure placement to htbp
\makeatletter
\def\fps@figure{htbp}
\makeatother


%\newcommand{\fuenteprincipal}{Hoefler Text}
\newcommand{\fuenteprincipal}{Adobe Jenson Pro}
\setmainfont{\fuenteprincipal}
\setromanfont{\fuenteprincipal}
\newfontfamily\headingfont[]{\fuenteprincipal}


%para bilbio
\usepackage{polyglossia,csquotes}
\setdefaultlanguage{spanish}
\usepackage[backend=biber, style=apa, refsection=subsection]{biblatex}
\DeclareLanguageMapping{spanish}{spanish-apa}
\addbibresource{/Users/juanespejo/Documents/Latex/BibTex/hume.bib}
\defbibheading{bibliography}{\section*{Bibliografía}}
\usepackage[left=2.5 cm,top=3cm,right=2.5cm, bottom=2.5cm]{geometry}



\begin{document}
{ \huge Philosophy and film}

{ \large \emph{Philosophy in and of Film}}

\subsection{Course description}\label{descripciuxf3n-del-seminario}

The main aim of the course is to develope language skills in English while at the same time studying philosophical problems around film. The student's philosophical study will be guided by two broad kinds of problems: philosophical problems in or motivated by determinate films and philosophical problems about film as a medium and a form of art. The student's language skills in English will be developed by the active use of the language in written or oral discussion, the writting of short texts and the comprehension of lectures, films and texts.


\emph{Professor:} \href{http://jcunisabana.github.io/}{Juan Camilo Espejo-Serna}

\emph{Email:} juan.espejo1@unisabana.edu.co

\emph{Time:} 

\emph{Room:} 

\emph{Website:} http://jcunisabana.github.io/

\subsection{Learning outcomes}
\begin{itemize}
\item
  The student produces short texts in English.
  \item
    The student plans philosophical essays in English.
\item
  The student understands key philosophical issues introduced by films and of film as a form of art.
\item
  The student understands the core elements of lectures, films and texts in English
\item
 The student uses ICTs to support their learning of English and philosophy.
\end{itemize}

\subsection{Assessment}
Se evaluará a los estudiantes por medio de controles de lectura,
protocolos, relatorías, y un ensayo argumentativo.

Unjustified late submissions will \textbf{always} be marked down 0.5 units per day.

\begin{longtable}[]{@{}clc@{}}
\toprule
Module & Activity & Percentage\tabularnewline
\midrule
\endhead
1 & Controles de lectura & 15\%\tabularnewline
1 & Reseña & 15\%\tabularnewline
2 & Protocolo & 15\%\tabularnewline
2 & Relatoría & 15\%\tabularnewline
3 & Abstract del ensayo argumentativo & 10\%\tabularnewline
3 & Ensayo argumentativo & 30\%\tabularnewline
\bottomrule
\end{longtable}

\paragraph{Controles de lectura}\label{controles-de-lectura}

Para cada lectura asignada, los estudiantes deben escribir un texto
corto con la tesis principal, tres afirmaciones/presuposiciones del
texto y tres preguntas/desafíos al texto. Los controles deberán ser
subidos a la plataforma virtual a más tardar el \emph{viernes a las 7 de
la mañana}. Todos los estudiantes empiezan con 5.0 en esta nota; por
cada vez que no se participe dentro del rango de tiempo especificado, la
nota será disminuida de acuerdo con los siguientes parámetros: primera
vez: -0.5; segunda vez: -1.0; tercera vez: -1.5; cuarta vez: -2.0.

\paragraph{Reseña}\label{reseuxf1a}

Resumen argumentativo: debe presentar el punto central del texto,
explicarlo y señalar las razones principales ofrecidas en su favor. En
clase se ofrecerán las instrucciones más precisas cuando llegue el
momento.

\paragraph{Protocolo}\label{protocolo}

Recuento de una sesión. Debe incluir los puntos centrales tanto de la
relatoría como de la discusión y  procurar seguir las mismas
instrucciones dadas para la reseña de un texto escrito. En clase se
ofrecerán las instrucciones más precisas.

\paragraph{Relatoría}\label{relatoruxeda}

Escrito para ser presentado oralmente en clase. Debe incluir una reseña
del texto asignado y presentar puntos para la discusión. En clase se
ofrecerán las instrucciones más precisas.

\paragraph{Abstract del ensayo
argumentativo}\label{abstract-del-ensayo-argumentativo}

Un texto argumentativo de mínimo 400 y máximo 500 palabras en donde se
presente un resumen de la tesis principal y la estrategia argumentativa
que usará.

\paragraph{Ensayo argumentativo}\label{ensayo-argumentativo}

Un texto argumentativo de 4-6 páginas en donde se defienda una tesis
filosófica. Se debe hacer uso de literatura secundaria de acuerdo con
las normas de citación. En clase se ofrecerán las instrucciones más
precisas cuando llegue el momento.

\nocite{*}
\printbibliography
\end{document}
